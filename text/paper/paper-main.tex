\documentclass{sig-alternate-br}
\setlength{\paperheight}{297mm}
\usepackage{comment} % for comment blocks
\usepackage{mathtools} % for equation
\usepackage[table]{xcolor} % for grey cells in tables
\usepackage{multicol}
\usepackage{algorithm} % for pseudocode
\usepackage{algpseudocode} % for pseudocode
\usepackage[pdfpagelabels=false,pageanchor=false,hidelinks]{hyperref} % to make references clickable
\usepackage[capitalise,noabbrev,nameinlink]{cleveref}
\usepackage[caption=false]{subfig} % for multiple figures in q1
\usepackage{svg} % for svg images
\usepackage{pgfplots} % graphs
\usepackage{tabu} % awesome table
\usepackage{float}
\usepackage[outline]{contour}
\usepackage{graphics}

\crefname{line}{line}{lines}

\makeatletter
\newcounter{algorithmicH}% New algorithmic-like hyperref counter
\let\oldalgorithmic\algorithmic
\renewcommand{\algorithmic}{%
  \stepcounter{algorithmicH}% Step counter
  \oldalgorithmic}% Do what was always done with algorithmic environment
\newcommand{\theHALG@line}{ALG@line.\thealgorithmicH.\arabic{ALG@line}}
\DeclareRobustCommand{\crefnosort}[1]{%
  \begingroup\@cref@sortfalse\cref{#1}\endgroup
}
\makeatother
\newcommand{\creflastconjunction}{, and\nobreakspace}

\pgfplotsset{compat=1.13}
\pdfsuppresswarningpagegroup=1

\contourlength{0.15em}

\newcommand\todo[1]{\textcolor{red}{Todo: #1 \typeout{^^J^^J LaTeX Warning: Todo on page \thepage, undefined on input line \the\inputlineno.^^J^^J}}}

%\def\NoNumber#1{{\def\alglinenumber##1{}\State #1}\addtocounter{ALG@line}{-1}}

\definecolor{jml}{rgb}{0.1,0.3,0.1}
\renewcommand\And{\textbf{ and }}
\newcommand\Old[1]{\textproc{\textbackslash{}old}(#1)}
\newcommand\Perm[2]{\textproc{Perm}(#1, #2)}
\algnewcommand\invariant[1]{\State{{\color{jml}\textbf{invariant} #1}}}
\algnewcommand\requires[1]{\State{{\color{jml}\textbf{requires} #1}}}
\algnewcommand\globalrequires[1]{\State{{\color{jml}\textbf{kernel requires} #1}}}
\algnewcommand\ensures[1]{\State{{\color{jml}\textbf{ensures} #1}}}
\algnewcommand\globalensures[1]{\State{{\color{jml}\textbf{kernel ensures} #1}}}
\algnewcommand\loopinv[1]{\State{{\color{jml}\textbf{loop\_invariant} #1}}}
\algnewcommand\context[1]{\State{{\color{jml}\textbf{context} #1}}}
\algnewcommand\JML[1]{\State{{\color{jml}#1}}}
\algnewcommand\JMLx[1]{\Statex{{\color{jml}#1}}}
\algnewcommand\Forall[3]{\textbf{(\textbackslash{}forall} #1; #2; #3)}
\algnewcommand\Forallx[3]{\textbf{(\textbackslash{}forall*} #1; #2; #3)}
\algnewcommand\Or{\textbf{or}}
\algdef{SE}[SUBALG]{Indent}{EndIndent}{}{\algorithmicend\ }%
\algtext*{Indent}
\algtext*{EndIndent}

%\DeclareCaptionLabelFormat{cont}{#1 #2\alph{ContinuedFloat}}
%\captionsetup[ContinuedFloat]{labelformat=cont}
%\renewcommand\theContinuedFloat{\alph{ContinuedFloat}}

\conferenceinfo{27$^{th}$ Twente Student Conference on IT,}{July 7, 2017, Enschede, The Netherlands.}
\CopyrightYear{2017}

\title{Edit distance on GPU clusters using MPI}

\newcommand\me{Antoine Veenstra}

\numberofauthors{1}
\author{
    \alignauthor \me\\
    \affaddr{University of Twente}\\
    \affaddr{P.O. Box 217, 7500AE Enschede}\\
    \affaddr{The Netherlands}\\
    \email{a.j.veenstra@student.utwente.nl}
}

\makeatletter
\hypersetup{
    pdftitle={\@title},
    pdfauthor={\me},
    pdftoolbar=false,        % show Acrobat’s toolbar?
    pdfmenubar=false,        % show Acrobat’s menu?
    pdffitwindow=false,     % window fit to page when opened
    pdfstartview={FitH},    % fits the width of the page to the window
    pdfkeywords={OpenCL, Edit distance problem, Levenshtein distance problem, GPU cluster, C++, case study, GPGPU program, MPI, Message Passing Interface}, % list of keywords
}
\makeatother

\begin{document}

\maketitle

\begin{abstract}
In this paper, we describe a verified implementation of the Levenshtein distance problem on a GPU cluster using MPI and OpenCL.
The implementation is based on an existing verified single GPU implementation.
The speed of the implementation is higher on a cluster, but the efficiency is affected by the overhead, which is caused by the extra communication between nodes.
\end{abstract}

\keywords{OpenCL, MPI, Message Passing Interface, C++, Edit distance problem, Levenshtein distance problem, GPU cluster, GPGPU program, case study}

\section{Introduction}
\todo{Improve}
As the amount of data increases the need for parallel computation does so too, but this is no small feat.
Multiple ways exist to process data in parallel, but one of the most efficient ways to do this is to use the Graphical Processing Unit (GPU).
A GPU enables the parallel execution of a single operation on multiple variables, unlike the Common Processing Unit (CPU) which only allows for the execution of a single operation on a single value.
Even if the CPU has multiple cores the amount of data processed in parallel is generally still inferior to that of a GPU.

To decrease the processing time even further a logical step is to increase the number of GPUs \cite{Cluster}.
One could add more GPUs to their computer, but this is not a scalable solution since most motherboards only support a limited amount of GPUs.
Another solution would be to make multiple devices work together, each containing at least one GPU.
This is called a GPU-cluster.

Various algorithms have already been implemented on a single GPU device and were verified \cite{Heus}.
The verification of a program is important since it can guarantee the result of an algorithm is always correct without fail.
If one wants to distribute a verified implementation over multiple nodes some steps have to be taken to ensure the implementation is still mathematically correct.
Those steps will be explored while distributing a verified implementation of the edit distance problem.

The edit distance problem is used in various fields of research \cite{Navarro:2001:GTA:375360.375365}.
Fields such as Computational Biology, Signal Processing, and Text Retrieval.
It is used to compare two strings or sequences of data, such as genome sequences.

The existing implementation of the edit distance problem uses a dynamic programming algorithm, which is well-suited for general-purpose computing on GPUs (GPGPU).
The implementation was written in C++ using OpenCL, which can run on most GPUs \cite{Kronos:conformant}.
An alternative for OpenCL would have been CUDA, which has been developed by NVIDIA and runs exclusively on NVIDIA GPUs.
OpenCL has been and will be chosen over CUDA in order to ensure compatibility with most GPUs.

To allow interaction between devices in a cluster a protocol is required.
One standard which has been around for years is the Message Passing Interface (MPI) \cite{MPI}.
This interface will be used to distribute the algorithm on multiple (multi-)GPU nodes.
A successful program using MPI and OpenCL has already been made \cite{Cluster}, so it might be possible to make a MPI-OpenCL implementation of the edit distance problem.

By implementing the edit distance problem on a GPU-cluster instead of a single GPU the processing time could be reduced as the performance of a cluster exceeds that of a single unit \cite{Cluster}.
The goal of this research is to implement the edit distance problem on a GPU-cluster using MPI and verify this implementation.
This goal gives us the research question mentioned in the following section.



\section{Research questions} \label{questions}
The research question of this paper is:

What are the steps required to distribute a verified implementation of an algorithm on a GPU-cluster?

A possible division in subquestions is:
\begin{enumerate}
    \item How can the algorithm be divided in separate processes?
    \item How can the algorithm be run on multiple devices using MPI?
    \item How can the verification of the implementation be guaranteed?
    \item What is the optimal number of GPUs when considering cost, efficiency, and the amount of data compared?
\end{enumerate}



\section{Background}
\subsection{OpenCL}
GPGPU programming is the use of GPUs to handle computation which traditionally is done by CPUs.
A CPU consists of one or more cores allowing Single Instructions streams and a Single Data stream (SISD).
A GPU on the other hand has a Single Instruction stream and Multiple Data streams (SIMD).
The number of cores on a GPU is generally much higher than a CPU has, so a GPU can process more data in parallel using its SIMD architecture.

One programming language allowing the developer to run programs on a GPU is OpenCL.
OpenCL allows a developer to run a kernel on a GPU or CPU \cite{OpenCL}.
It is a low level programming language which can run on most GPUs and CPUs and allows general purpose parallel programming across both CPUs and GPUs.
The traditional CPU-based programming models do not allow the same complex vector operations on GPUs as OpenCL offers without the need to translate their algorithms to a 3d graphics API such as OpenGL.
As mentioned before OpenCL is preferred over CUDA since the support of CUDA for GPUs and CPUs is limited to NVIDIA GPUs \cite{CUDA}.

In the OpenCL architecture one CPU-based program called the "Host" controls multiple GPUs and CPUs called "Compute Devices".
Each of those compute devices consist of one or more "Compute Units", which each contain one or more "Processing Elements".
These processing elements execute the OpenCL kernels provided by the host program.
After such kernel has finished running the results are returned to the host program.
The kernel on every processing element is the same, so the only way to change the outcome of the program is to modify the input of the program \cite{OpenCL}.

The memory hierarchy used in OpenCL is not equal to that of the physical memory configuration on GPUs.
This is to prevent having to take into account every type of architecture, which would be tedious work as the amount of types is quite large.
Each of the architectural devices discussed above have their own memory, which is inaccessible to components of the same type.
Every processing element can access its own private memory, the memory of its compute unit, the memory of its compute device.
The host memory can be accessed, but it is generally slower than the on-board memory \cite{OpenCL}.

The architecture and memory hierarchy already enforce the division of the algorithm if one wants to use every component of a GPU.

\subsection{MPI}
MPI is a standard specification in communication between computers which enables parallel computing.
An implementation of this specification is freely available and will be used in this project.
The implementation allows the transmission of multiple datatypes and messages between nodes.
It also provides a way to identify all the connected processes and assign an identifier to each process \cite{MPI}.
These features could help dividing the workload over all nodes.

\subsection{Edit distance}
The edit distance problem is way of measuring how much two strings differ from each other \cite{Navarro:2001:GTA:375360.375365}.
The distance is measured by operations like inserting, removing, replacing, and rearranging characters.
The complexity of the algorithm depends on what operations are allowed and the cost of these operations in the implementation.
For this project only inserting, removing, and replacing are considered.
The operations costs are $C_i$ (insertion), $C_d$ (deletion), and $C_r$ (replacement).

An example input is:
\begin{align*}
S_1 &= \text{Saturday}\\
S_2 &= \text{Sunday}\\
C_i &= 2\\
C_d &= 3\\
C_r &= 4
\end{align*}



%\section{Related work}

\subsection{Edit distance problem on GPU}
As mentioned before, the edit distance problem has already been implemented on a single GPU by De Heus~\cite{Heus}.
His implementation also uses the Levenshtein distance.
This implementation was used as base in~\cref{originalalg}.\todo{fix}
The new implementation could be compared to this single node implementation to calculate the difference in time required to compute the edit distance of two strings.\todo{fix}

%\balancecolumns
\subsection{Benchmark on a GPU cluster}
This is an MPI-OpenCL implementation of the LINPACK benchmark, which was run on a cluster containing 49 nodes, each node containing two eight-core CPUs and four GPUs~\cite{Cluster}.
Their implementation achieves 93.69 Tflops, which is 46 percent of the theoretical peak.
It shows a successful implementation of MPI in combination with OpenCL, which is required in the implementation of the edit distance algorithm.


\section{Dividing the algorithm}
The algorithm de Heus uses a dynamic programming solution\cite{Heus}.
In his paper he describes a way to distribute the computation on multiple work-groups of a GPU.
The dynamic programming algorithm fills a matrix with the following rules\cite{Jordan}:

\begin{equation} \label{eq1}
\begin{split}
H_{(-1,j)} & = j \\
H_{(i,-1)} & = i \\
H_{(i,j)} & = \min \begin{cases}
          \operatorname{H}_{(i-1,j)} + 1 \\
          \operatorname{H}_{(i,j-1)} + 1 \\
          \operatorname{H}_{(i-1,j-1)} + Score
\end{cases}
\end{split}
\end{equation}

Where $Score$ is zero if the characters of the compared sequences at index $i$ and $j$ are equal; otherwise, the $Score$ is one.

The value of $H_{(i,j)}$ depends on the cells $H_{(i-1,j)}$, $H_{(i,j-1)}$, and $H_{(i-1,j-1)}$.
This limits the use of parallelism to speed up the computation, but it leaves an opening none the less.
There is no dependency between cells $H_{(a,b)}$ if $a + b$ is constant.
The grey cells in table \ref{diagonal} are such a group of cells which can be calculated in parallel.
Each diagonal is based on the previous two diagonals, because of the dependencies previously mentioned\cite{Meyers}.
There is no need to save diagonals prior to those two diagonals, so the implementation can discard the previous diagonals to save memory.

{\newcommand\C[0]{\cellcolor{gray}}
\begin{table}[h]
\centering \large
\begin{tabular}{|c|c||c|c|c|c|c|c|} \hline
           &            & \textbf{k} & \textbf{i} & \textbf{t} & \textbf{t} & \textbf{e} & \textbf{n} \\ \hline
           & \textbf{0} & \textbf{1} & \textbf{2} & \textbf{3} & \textbf{4} & \textbf{5} & \textbf{6} \\ \hline \hline
\textbf{s} & \textbf{1} & 1          & 2          & 3          & 4          & \C         &            \\ \hline
\textbf{i} & \textbf{2} & 2          & 1          & 2          & \C         &            &            \\ \hline
\textbf{t} & \textbf{3} & 3          & 2          & \C         &            &            &            \\ \hline
\textbf{t} & \textbf{4} & 4          & \C         &            &            &            &            \\ \hline
\textbf{i} & \textbf{5} & \C         &            &            &            &            &            \\ \hline
\textbf{n} & \textbf{6} &            &            &            &            &            &            \\ \hline
\textbf{g} & \textbf{7} &            &            &            &            &            &            \\ \hline
\end{tabular}
\caption{Example matrix} \label{diagonal}
\end{table}
}

With larger sequences the diagonal becomes too large to calculate in one iteration on a GPU.
Dividing the matrix vertically allows one to split the calculation in manageable parts.
Each part requires the right most column of the previous part due to the dependencies of each cell.
This means the other columns can be discarded to save memory.

Each of the parts mentioned above can be split in blocks.
Blocks $A$ to $C$ and $D$ to $F$ in table \ref{division} are such a partitioning.
The dependencies of the individual cells are inherited by the individual blocks.
Just like the cells the blocks can also be calculated in parallel if they are not dependent of one another.
Block $D$ and $c$ are such blocks as they only require block $B$.
With larger sequences the number of independent blocks becomes more significant.
As a result the calculation of multiple blocks in parallel becomes more attractive.

Blocks $B$ and $E$ are constructed diagonally to optimise the amount of threads at work at any given time.
If the blocks where squares \todo{finish this}


\begin{table}[h]
    \centering
    \includesvg[width=0.95\linewidth]{cols}
    \caption{Division of the matrix} \label{division}
\end{table}


\begin{comment}
\begin{algorithm}
\caption{Parallel Levenshtein Algorithm} \label{pseudo}
\begin{algorithmic}[1]
\Procedure{Levenshtein}{$col, row, id, n, a, b, d1, d2, out$}\Comment{The Levenshtein distance between sequences $a$ and $b$}
    \Require $n is even$
    \Require $out is int[n]$
    \State $col\gets col + id$
    \State $row\gets row - id$
    \For{$row \dots row + n$}
        %\If{$row \get $}
        \If{$a[col] \not= b[row]$}
            \State $cost\gets 1$
        \Else
            \State $cost\gets 0$
        \EndIf
        
        
    \EndFor
\EndProcedure
\end{algorithmic}
\end{algorithm}
\end{comment}

OpenCL does not support the creation of arrays with multiple dimensions, so 
%3 algorithms

%1 fill_row

%2 fill_defaults

%3 main


\section{Using MPI to divide the algorithm}



\section{Verifying the implementation} \label{q3}
In \crefrange{block}{MPIalg} verification has been added in the form of \textit{ensures} and \textit{requires}.
The verification guarantees that the work-items do not write to the same memory location.
This is required to ensure the determinism of the algorithm, so that the result is consistent.
As explained in \cref{backver}, permission-based separation logic is used to state which work-item has access to what resources.

Most of the lines are trivial or explained in \cref{algorithms}.
Non-trivial lines will be explained in this section.
The OpenCL algorithms are executed as a single group, so the work-group verification is equal to the kernel verification
Therefore, the work-group verification has been left out.

\subsection{OpenCL Algorithm \ref*{block}} \label{ver:block}
In \cref{block}, \crefrange{block:barrier:requires}{block:barrier} describe how the barrier distributes the read and write permissions among the work-items.
\Cref{block:barrier:requires} reclaims all permissions the work-items have on $d$.
The following line redistributes the permissions, and \cref{block:barrier:global} describes how the permissions are distributed over the kernel.
Since the difference of the $x$s between two consecutive threads is 2, no threads should have read permissions on a cell with write permissions of another work-item.

\Cref{block:firstcell,block:lastcell} refer to the fact that the first and last cells in the array of diagonals are not edited, as \Cref{singlevar} illustrates.

\subsection{OpenCL Algorithm \ref*{begin}} \label{ver:begin}
The explanation of the previous section also holds up for \cref{begin}, but a few line have been added.
\Cref{begin:dall,begin:did,begin:dwidth} are required in this algorithm to enable the writes in \crefrange{begin:fillbegin}{begin:fillend}.
\Cref{begin:barrier:did,begin:barrier:dwidth} reclaim those permissions.
\Crefrange{begin:barrier:requires}{begin:barrier:global} of \cref{begin} are equivalent to \crefrange{block:barrier:requires}{block:barrier:global} from \cref{block}.
The if statement on \cref{begin:ifbegin} helps with the enforcement of permissions, as less work-items process cells at the same time.

\subsection{OpenCL Algorithm \ref*{fill_column}} \label{ver:fill_column}
Verification of \cref{fill_column} is trivial, as no work-item requires read permissions and no write permissions overlap.
\Cref{fill_column:dall,fill_column:did} describe what write permissions are required.

\begin{figure*}[!htb]
    \centering
    \subfloat[][Overview]{\label{result_graph}
\begin{tikzpicture}
\begin{axis}[
    %title={Average time to compute edit distance},
    xlabel={Number of cells},
    ylabel={Average time (s)},
    xmin=0, xmax=125000000000,
    ymin=0, ymax=150,
    legend pos=south east,
    ymajorgrids=true,
    grid style=dashed,
    scaled x ticks={real:10000000000},
    width=0.45\linewidth,
]


\addplot[
    color=blue,
    smooth,
    ]
    coordinates {
    (16777216, 0.09921685714285713)(33554432, 0.12080128571428571)(67108864, 0.160687)(134217728, 0.2424785714285714)(268435456, 0.38340228571428575)(536870912, 0.6664877142857142)(1000000000, 1.152977142857143)(1500000000, 1.6785985714285712)(2147483648, 2.357917142857143)(3000000000, 3.2444357142857143)(4294967296, 4.582278571428572)(6000000000, 6.368568571428573)(7000000000, 7.4005728571428575)(8589934592, 9.062895000000001)(10000000000, 10.52197142857143)(11000000000, 11.564114285714284)(12000000000, 12.593114285714284)(13000000000, 13.643714285714285)(14000000000, 14.677214285714285)(17179869184, 17.97048571428571)(25769803776, 26.88624285714286)(34359738368, 35.8241)(42949672960, 44.70372857142858)(51539607552, 53.59438571428571)(60129542144, 62.536071428571425)(68719476736, 71.40738571428571)(77309411328, 80.32997142857143)(85899345920, 89.26152857142857)(94489280512, 98.17172857142855)(100000000000, 103.8337142857143)(103079215104, 107.02371428571429)(111669149696, 115.95128571428572)(120259084288, 124.80914285714286)
    };

\addplot[
    color=red,
    smooth,
    ]
    coordinates {
    (16777216, 0.5445155714285714)(33554432, 0.626533)(67108864, 0.8027252857142858)(134217728, 1.12474)(268435456, 1.7438942857142856)(536870912, 3.0209571428571427)(1000000000, 5.162437142857143)(1500000000, 7.543924285714286)(2147483648, 10.5999)(3000000000, 14.472014285714282)(4294967296, 20.55514285714286)(6000000000, 28.49077142857143)(7000000000, 33.10725714285714)(8589934592, 40.61537142857143)(10000000000, 47.16012857142857)(11000000000, 51.74827142857142)(12000000000, 56.35255714285715)(13000000000, 61.09289999999999)(14000000000, 65.73704285714287)(17179869184, 80.55284285714286)(25769803776, 120.55142857142857)(34359738368, 160.5607142857143)(42949672960, 200.45)(51539607552, 240.47842857142857)(60129542144, 280.283)(68719476736, 320.3165714285714)(77309411328, 360.40957142857144)(85899345920, 400.24628571428576)(94489280512, 440.2641428571429)(100000000000, 465.98900000000003)(103079215104, 480.2848571428571)(111669149696, 520.3457142857143)(120259084288, 560.304)
    };

\addplot[
    color=green,
    smooth,
    ]
    coordinates {
    (16777216, 0.496333)(33554432, 0.5080268571428571)(67108864, 0.6234567142857143)(134217728, 0.6975642857142856)(268435456, 0.7566615714285714)(536870912, 1.0175524285714288)(1000000000, 1.4202885714285713)(1500000000, 1.9164114285714287)(2147483648, 2.5176842857142856)(3000000000, 3.249014285714286)(4294967296, 4.516901428571428)(6000000000, 6.0832242857142855)(7000000000, 7.0254957142857135)(8589934592, 8.529228)(10000000000, 9.85383)(11000000000, 10.788914285714286)(12000000000, 11.679885714285714)(13000000000, 12.648942857142856)(14000000000, 13.559485714285714)(17179869184, 16.574385714285714)(25769803776, 24.604942857142856)(34359738368, 32.70814285714285)(42949672960, 40.747)(51539607552, 48.7744)(60129542144, 56.66547142857142)(68719476736, 64.66221428571428)(77309411328, 72.89688571428572)(85899345920, 81.01222857142857)(94489280512, 89.02959999999999)(100000000000, 94.25682857142858)(103079215104, 97.06402857142858)(111669149696, 105.15614285714285)(120259084288, 113.20742857142857)
    };

\addplot[
    color=black,
    mark=none,
] coordinates {(0,0) (0,12) (10000000000,12) (10000000000,0) (0,0)};

\pgfplotsset{
    after end axis/.code={
        \node[above] at (axis cs:10000000000,12){\contour{white}{Fig. \ref{result_graph_zoom}}};
    }
}


\legend{Node 1, Node 2, Nodes 1 and 2}
\end{axis}
\end{tikzpicture}
}
    \hspace{0.05\linewidth}
    \subfloat[][Zoomed in]{\label{result_graph_zoom}
\begin{tikzpicture}
\begin{axis}[
    %title={Average time to compute edit distance},
    xlabel={Number of cells},
    ylabel={Average time (s)},
    xmin=0, xmax=10000000000,
    ymin=0, ymax=12,
    legend pos=south east,
    ymajorgrids=true,
    grid style=dashed,
    scaled x ticks={real:10000000000},
    width=0.45\linewidth,
]


\addplot[
    color=blue,
    smooth,
    ]
    coordinates {
    (16777216, 0.09921685714285713)(33554432, 0.12080128571428571)(67108864, 0.160687)(134217728, 0.2424785714285714)(268435456, 0.38340228571428575)(536870912, 0.6664877142857142)(1000000000, 1.152977142857143)(1500000000, 1.6785985714285712)(2147483648, 2.357917142857143)(3000000000, 3.2444357142857143)(4294967296, 4.582278571428572)(6000000000, 6.368568571428573)(7000000000, 7.4005728571428575)(8589934592, 9.062895000000001)(10000000000, 10.52197142857143)(11000000000, 11.564114285714284)(12000000000, 12.593114285714284)(13000000000, 13.643714285714285)(14000000000, 14.677214285714285)(17179869184, 17.97048571428571)(25769803776, 26.88624285714286)(34359738368, 35.8241)(42949672960, 44.70372857142858)(51539607552, 53.59438571428571)(60129542144, 62.536071428571425)(68719476736, 71.40738571428571)(77309411328, 80.32997142857143)(85899345920, 89.26152857142857)(94489280512, 98.17172857142855)(100000000000, 103.8337142857143)(103079215104, 107.02371428571429)(111669149696, 115.95128571428572)(120259084288, 124.80914285714286)
    };

\addplot[
    color=red,
    smooth,
    ]
    coordinates {
    (16777216, 0.5445155714285714)(33554432, 0.626533)(67108864, 0.8027252857142858)(134217728, 1.12474)(268435456, 1.7438942857142856)(536870912, 3.0209571428571427)(1000000000, 5.162437142857143)(1500000000, 7.543924285714286)(2147483648, 10.5999)(3000000000, 14.472014285714282)(4294967296, 20.55514285714286)(6000000000, 28.49077142857143)(7000000000, 33.10725714285714)(8589934592, 40.61537142857143)(10000000000, 47.16012857142857)(11000000000, 51.74827142857142)(12000000000, 56.35255714285715)(13000000000, 61.09289999999999)(14000000000, 65.73704285714287)(17179869184, 80.55284285714286)(25769803776, 120.55142857142857)(34359738368, 160.5607142857143)(42949672960, 200.45)(51539607552, 240.47842857142857)(60129542144, 280.283)(68719476736, 320.3165714285714)(77309411328, 360.40957142857144)(85899345920, 400.24628571428576)(94489280512, 440.2641428571429)(100000000000, 465.98900000000003)(103079215104, 480.2848571428571)(111669149696, 520.3457142857143)(120259084288, 560.304)
    };

\addplot[
    color=green,
    smooth,
    ]
    coordinates {
    (16777216, 0.496333)(33554432, 0.5080268571428571)(67108864, 0.6234567142857143)(134217728, 0.6975642857142856)(268435456, 0.7566615714285714)(536870912, 1.0175524285714288)(1000000000, 1.4202885714285713)(1500000000, 1.9164114285714287)(2147483648, 2.5176842857142856)(3000000000, 3.249014285714286)(4294967296, 4.516901428571428)(6000000000, 6.0832242857142855)(7000000000, 7.0254957142857135)(8589934592, 8.529228)(10000000000, 9.85383)(11000000000, 10.788914285714286)(12000000000, 11.679885714285714)(13000000000, 12.648942857142856)(14000000000, 13.559485714285714)(17179869184, 16.574385714285714)(25769803776, 24.604942857142856)(34359738368, 32.70814285714285)(42949672960, 40.747)(51539607552, 48.7744)(60129542144, 56.66547142857142)(68719476736, 64.66221428571428)(77309411328, 72.89688571428572)(85899345920, 81.01222857142857)(94489280512, 89.02959999999999)(100000000000, 94.25682857142858)(103079215104, 97.06402857142858)(111669149696, 105.15614285714285)(120259084288, 113.20742857142857)
    };


\legend{Node 1, Node 2, Nodes 1 and 2}
\end{axis}
\end{tikzpicture}
}
    \caption{Average time to compute edit distance} \label{result_graph_total}
    %\floatpagefraction
%\end{figure*}
\vspace*{\floatsep}
%\begin{figure*}[h]
    \centering
    \subfloat[][Overview]{\label{result_graph_cell}
\begin{tikzpicture}
\begin{axis}[
    %title={Average time per cell},
    xlabel={Number of cells},
    ylabel={Average time (s)},
    xmin=0, xmax=20000000000,
    ymin=7e-10, ymax=7e-09,
    legend pos=north east,
    ymajorgrids=true,
    grid style=dashed,
    scaled x ticks={real:10000000000},
]


\addplot[
    color=blue,
    smooth,
    ]
    coordinates {
    (16777216, 5.9137855257306775e-09)(33554432, 3.600158861705235e-09)(67108864, 2.3944228887557983e-09)(134217728, 1.8066061394555226e-09)(268435456, 1.4282848153795516e-09)(536870912, 1.2414301080363135e-09)(1000000000, 1.152977142857143e-09)(1500000000, 1.119065714285714e-09)(2147483648, 1.0979907321078438e-09)(3000000000, 1.0814785714285714e-09)(4294967296, 1.0668948691870486e-09)(6000000000, 1.0614280952380955e-09)(7000000000, 1.057224693877551e-09)(8589934592, 1.0550598381087185e-09)(10000000000, 1.0521971428571429e-09)(11000000000, 1.0512831168831168e-09)(12000000000, 1.0494261904761902e-09)(13000000000, 1.0495164835164835e-09)(14000000000, 1.0483724489795918e-09)(17179869184, 1.0460199389074528e-09)(25769803776, 1.0433235383104711e-09)(34359738368, 1.042618532665074e-09)(42949672960, 1.0408397897013597e-09)(51539607552, 1.0398679435075748e-09)(60129542144, 1.0400224115927607e-09)(68719476736, 1.0391142235935798e-09)(77309411328, 1.0390710529117357e-09)(85899345920, 1.0391409575406993e-09)(94489280512, 1.0389721250863041e-09)(100000000000, 1.0383371428571429e-09)(103079215104, 1.0382666784737793e-09)(111669149696, 1.038346634051957e-09)(120259084288, 1.0378354666184407e-09)
    };

\addplot[
    color=red,
    smooth,
    ]
    coordinates {
    (16777216, 3.245565720966884e-08)(33554432, 1.8672138452529907e-08)(67108864, 1.1961538876805988e-08)(134217728, 8.379966020584107e-09)(268435456, 6.4965124641145975e-09)(536870912, 5.626971168177468e-09)(1000000000, 5.162437142857143e-09)(1500000000, 5.029282857142857e-09)(2147483648, 4.935963079333305e-09)(3000000000, 4.824004761904761e-09)(4294967296, 4.785867141825813e-09)(6000000000, 4.748461904761905e-09)(7000000000, 4.729608163265305e-09)(8589934592, 4.728251535977636e-09)(10000000000, 4.7160128571428574e-09)(11000000000, 4.704388311688311e-09)(12000000000, 4.696046428571429e-09)(13000000000, 4.699453846153845e-09)(14000000000, 4.695503061224491e-09)(17179869184, 4.688792562644397e-09)(25769803776, 4.678011117946534e-09)(34359738368, 4.6729318065834904e-09)(42949672960, 4.667090252041816e-09)(51539607552, 4.665895609096402e-09)(60129542144, 4.661319378231253e-09)(68719476736, 4.661219593669687e-09)(77309411328, 4.661910694151644e-09)(85899345920, 4.659480016146388e-09)(94489280512, 4.6594083526885355e-09)(100000000000, 4.6598900000000005e-09)(103079215104, 4.659376351074094e-09)(111669149696, 4.6597087530644385e-09)(120259084288, 4.659140748637063e-09)
    };

\addplot[
    color=green,
    smooth,
    ]
    coordinates {
    (16777216, 2.9583752155303957e-08)(33554432, 1.514038017817906e-08)(67108864, 9.290228996958051e-09)(134217728, 5.197258932249886e-09)(268435456, 2.8187840112618036e-09)(536870912, 1.8953390951667517e-09)(1000000000, 1.4202885714285714e-09)(1500000000, 1.277607619047619e-09)(2147483648, 1.1723881055201802e-09)(3000000000, 1.083004761904762e-09)(4294967296, 1.0516730669353688e-09)(6000000000, 1.0138707142857144e-09)(7000000000, 1.003642244897959e-09)(8589934592, 9.929328225553036e-10)(10000000000, 9.85383e-10)(11000000000, 9.808103896103896e-10)(12000000000, 9.733238095238094e-10)(13000000000, 9.729956043956044e-10)(14000000000, 9.68534693877551e-10)(17179869184, 9.647562235061612e-10)(25769803776, 9.547974470825421e-10)(34359738368, 9.519322442689111e-10)(42949672960, 9.487150236964226e-10)(51539607552, 9.463479121526082e-10)(60129542144, 9.423898704045889e-10)(68719476736, 9.409590607642063e-10)(77309411328, 9.429238234010953e-10)(85899345920, 9.43106466106006e-10)(94489280512, 9.422190487384794e-10)(100000000000, 9.425682857142859e-10)(103079215104, 9.416450103301378e-10)(111669149696, 9.416758625225706e-10)(120259084288, 9.41362802167328e-10)
    };


\legend{Node 1, Node 2, Nodes 1 and 2}
\end{axis}
\end{tikzpicture}
}
    \hspace{0.05\linewidth}
    \subfloat[][Zoomed in]{\label{result_graph_cell_zoom}
\begin{tikzpicture}
\begin{axis}[
    %title={Average time per cell},
    xlabel={Number of cells},
    ylabel={Average time (s)},
    xmin=0, xmax=15000000000,
    ymin=9e-10, ymax=1.4e-09,
    legend pos=north east,
    ymajorgrids=true,
    grid style=dashed,
    scaled x ticks={real:10000000000},
]


\addplot[
    color=blue,
    smooth,mark=diamond,
    ]
    coordinates {
    (16777216, 5.9137855257306775e-09)(33554432, 3.600158861705235e-09)(67108864, 2.3944228887557983e-09)(134217728, 1.8066061394555226e-09)(268435456, 1.4282848153795516e-09)(536870912, 1.2414301080363135e-09)(1000000000, 1.152977142857143e-09)(1500000000, 1.119065714285714e-09)(2147483648, 1.0979907321078438e-09)(3000000000, 1.0814785714285714e-09)(4294967296, 1.0668948691870486e-09)(6000000000, 1.0614280952380955e-09)(7000000000, 1.057224693877551e-09)(8589934592, 1.0550598381087185e-09)(10000000000, 1.0521971428571429e-09)(11000000000, 1.0512831168831168e-09)(12000000000, 1.0494261904761902e-09)(13000000000, 1.0495164835164835e-09)(14000000000, 1.0483724489795918e-09)(17179869184, 1.0460199389074528e-09)(25769803776, 1.0433235383104711e-09)(34359738368, 1.042618532665074e-09)(42949672960, 1.0408397897013597e-09)(51539607552, 1.0398679435075748e-09)(60129542144, 1.0400224115927607e-09)(68719476736, 1.0391142235935798e-09)(77309411328, 1.0390710529117357e-09)(85899345920, 1.0391409575406993e-09)(94489280512, 1.0389721250863041e-09)(100000000000, 1.0383371428571429e-09)(103079215104, 1.0382666784737793e-09)(111669149696, 1.038346634051957e-09)(120259084288, 1.0378354666184407e-09)
    };

\addplot[
    color=green,
    smooth,mark=square,
    ]
    coordinates {
    (16777216, 2.9583752155303957e-08)(33554432, 1.514038017817906e-08)(67108864, 9.290228996958051e-09)(134217728, 5.197258932249886e-09)(268435456, 2.8187840112618036e-09)(536870912, 1.8953390951667517e-09)(1000000000, 1.4202885714285714e-09)(1500000000, 1.277607619047619e-09)(2147483648, 1.1723881055201802e-09)(3000000000, 1.083004761904762e-09)(4294967296, 1.0516730669353688e-09)(6000000000, 1.0138707142857144e-09)(7000000000, 1.003642244897959e-09)(8589934592, 9.929328225553036e-10)(10000000000, 9.85383e-10)(11000000000, 9.808103896103896e-10)(12000000000, 9.733238095238094e-10)(13000000000, 9.729956043956044e-10)(14000000000, 9.68534693877551e-10)(17179869184, 9.647562235061612e-10)(25769803776, 9.547974470825421e-10)(34359738368, 9.519322442689111e-10)(42949672960, 9.487150236964226e-10)(51539607552, 9.463479121526082e-10)(60129542144, 9.423898704045889e-10)(68719476736, 9.409590607642063e-10)(77309411328, 9.429238234010953e-10)(85899345920, 9.43106466106006e-10)(94489280512, 9.422190487384794e-10)(100000000000, 9.425682857142859e-10)(103079215104, 9.416450103301378e-10)(111669149696, 9.416758625225706e-10)(120259084288, 9.41362802167328e-10)
    };


\legend{Node 1, Nodes 1 and 2}
\end{axis}
\end{tikzpicture}
}
    \caption{Average time per cell} \label{result_graph_cell_total}
\end{figure*}


\subsection{MPI Algorithm \ref*{MPIalg}} \label{ver:MPIalg}
As the nodes in a cluster only communicate the right most columns of blocks, no concurrent variable manipulation can occur.
Therefore, permission-based separation logic is not required for this algorithm.
There are no circular dependencies, so no deadlocks can occur.
This cuts down the need for verification even further, leaving only trivial \textit{requires} and \textit{ensures} statements.

\subsection{Verification with VerCors} \label{ver:vercors}
Unfortunately, OpenCL is not sufficiently supported by VerCors to do automatic verification.
Rewriting the kernel in PVL, the native language of VerCors, does not help, as there is no support for kernel arguments.
In further research support could be built into VerCors, but in the mean time the manual verification provided in \crefrange{block}{MPIalg} will have to do.


\section{Testing the performance of the implementation}


\section{Conclusion}
The steps required to distribute the verified implementation of the edit distance are described in \crefrange{q1}{q3}.
The first step is explained in \cref{q1}, which is the division of the algorithm as shown in \cref{division}.
The second step is explained in explained in \cref{q2}, which is the distribution of the pillars among nodes and the communication between the nodes.
Finally, \cref{q3} explains the verification of the new algorithms used in the previous two steps.
The three steps answer the first three subquestions mentioned in \cref{questions}.
Together they answer the research question mentioned in the same section.

The final subquestion is answered in \Cref{testing}, which discusses the performance of the implementation.
%It concludes that the speed of the implementation on a cluster is superior to the speed of the implementation o

\subsection{Future work on the implementation} \label{future}
There are certain optimisations possible for the MPI algorithm.
As mentioned in \cref{testing}, one node can be the bottleneck of a cluster.
This can be solved by either using the same hardware for every node, or to dynamically divide the work load between nodes.
The dynamic division should allow nodes to negotiate what their tasks will be.

Another optimisation is the use of the CPU parallel to the GPU.
The OpenCL algorithms can be converted to a CPU implementation so that it can help solving the edit distance problem.

The OpenCL algorithms could also be optimised.
The memory used could be limited by using the algorithm of Meyers \cite{Meyers}.
The advantage is that the columns shared between nodes takes less space, so the overhead for sending and receiving the columns should be reduced.
However a disadvantage is that the algorithm of Meyers requires extra operations per cell and thus reduces the performance of the algorithms.
Therefore, the overall performance of the implementation is not quaranteed to be better than the current implementation.


%\newpage

%\balancecolumns
%\begin{multicols*}{2}
\bibliographystyle{abbrv}
\bibliography{sigproc}
%\end{multicols*}

\end{document}
