\section{Introduction}
In recent history the amount of available data has increased, as faster computers acquire more information at a higher rate.
As the amount of data increases the need for parallel computation does so too to process said data.
This, however, is no small feat.
Multiple ways exist to process data in parallel, but one of the most efficient ways to do this is to use the Graphical Processing Unit (GPU).
A GPU enables the parallel execution of a single operation on multiple variables, unlike the Common Processing Unit (CPU) which only allows for the execution of a single operation on a single value.
Even if the CPU has multiple cores the maximum amount of data processed in parallel is generally still inferior to that of a GPU.

To decrease the processing time even further a logical step is to increase the number of GPUs \cite{Cluster}.
One could add more GPUs to their computer, but this is not a scalable solution since most motherboards only support a limited amount of GPUs.
Another solution would be to make multiple devices work together, each containing at least one GPU.
This is called a GPU cluster.

Various algorithms have already been implemented on a single GPU device and were verified \cite{Heus}.
The verification of a program is important since it can guarantee the outcome of an algorithm is always correct.
If one wants to distribute a verified implementation over multiple nodes additional steps have to be taken to ensure the implementation is still mathematically correct.
Those steps will be explored while distributing a verified implementation of the edit distance problem, which has been developed by De Heus \cite{Heus}.

The edit distance problem is used in various fields of research \cite{Navarro:2001:GTA:375360.375365}.
Fields such as Computational Biology, Signal Processing, and Text Retrieval.
It is used to compare two strings or sequences of data, such as genome sequences.
\todo{\Cref{bed} will explain what the problem goes. }

The existing implementation of the edit distance problem uses a dynamic programming algorithm, which is well-suited for general-purpose computing on GPUs (GPGPU).
The implementation was written in C++ using OpenCL, which can run on most GPUs \cite{Kronos:conformant}.
An alternative for OpenCL would have been CUDA, which has been developed by NVIDIA and runs exclusively on NVIDIA GPUs.
OpenCL has been and will be chosen over CUDA in order to ensure compatibility with most GPUs.

To allow interaction between devices in a cluster a protocol is required.
One standard which has been around for years is the Message Passing Interface (MPI) \cite{MPI}.
This interface will be used to distribute the algorithm on multiple (multi-)GPU nodes.
A successful program using MPI and OpenCL has already been made \cite{Cluster}, so it should be possible to make a MPI-OpenCL implementation of the edit distance problem.

By implementing the edit distance problem on a GPU cluster instead of a single GPU the processing time could be reduced as the performance of a cluster exceeds that of a single unit \cite{Cluster}.
The goal of this research is to implement the edit distance problem on a GPU cluster using MPI and verify this implementation.
This goal results in the research question mentioned in the following section.

