\section{Dividing the algorithm in multiple processes}
The algorithm de Heus uses a dynamic programming solution\cite{Heus:GPGPU}.
In his paper he describes a way to distribute the computation on multiple work-groups of a GPU.
The dynamic programming algorithm fills a matrix with the following rules\cite{Jordan}:

\begin{equation} \label{eq1}
\begin{split}
H_{(-1,j)} & = j \\
H_{(i,-1)} & = i \\
H_{(i,j)} & = \min \begin{cases}
          \operatorname{H}_{(i-1,j)} + 1 \\
          \operatorname{H}_{(i,j-1)} + 1 \\
          \operatorname{H}_{(i-1,j-1)} + Score
\end{cases}
\end{split}
\end{equation}

Where $Score$ is zero if the characters of the compared sequences at index $i$ and $j$ are equal; otherwise, the $Score$ is one.
%Note that if the characters are equal the value of $H_{(i-1,j-1)} + Score$ is always lower than or equal to 

The value of $H_{(i,j)}$ is depends on the cells $H_{(i-1,j)}$, $H_{(i,j-1)}$, and $H_{(i-1,j-1)}$.
This limits the use of parallelism to speed up the computation, but it leaves an opening none the less.
There is no dependency between cells $H_{(a,b)}$ if $a + b$ is constant.
The grey cells in table \ref{diagonal} are such a group of cells which can be calculated in parallel.

{\newcommand\C[0]{\cellcolor{gray}}

\begin{table}
\centering \caption{Example matrix} \label{diagonal}
\large
\begin{tabular}{|c|c||c|c|c|c|c|c|} \hline
           &            & \textbf{k} & \textbf{i} & \textbf{t} & \textbf{t} & \textbf{e} & \textbf{n} \\ \hline
           & \textbf{0} & \textbf{1} & \textbf{2} & \textbf{3} & \textbf{4} & \textbf{5} & \textbf{6} \\ \hline \hline
\textbf{s} & \textbf{1} & 1          & 2          & 3          & 4          & \C         &            \\ \hline
\textbf{i} & \textbf{2} & 2          & 1          & 2          & \C         &            &            \\ \hline
\textbf{t} & \textbf{3} & 3          & 2          & \C         &            &            &            \\ \hline
\textbf{t} & \textbf{4} & 4          & \C         &            &            &            &            \\ \hline
\textbf{i} & \textbf{5} & \C         &            &            &            &            &            \\ \hline
\textbf{n} & \textbf{6} &            &            &            &            &            &            \\ \hline
\textbf{g} & \textbf{7} &            &            &            &            &            &            \\ \hline
\end{tabular}
\end{table}
}

With larger sequences the diagonal becomes too large to calculate in one iteration.
At that point the use of multiple work-groups can be justified.
These groups will have to share some data between them.

\begin{algorithm}
\caption{Parallel Levenshtein Algorithm} \label{pseudo}
\begin{algorithmic}[1]
\Procedure{Levenshtein}{$col, row, id, n, a, b, d1, d2, out$}\Comment{The Levenshtein distance between sequences $a$ and $b$}
    \Require $n is even$
    \Require $out is int[n]$
    \State $col\gets col + id$
    \State $row\gets row - id$
    \For{$row \dots row + n$}
        %\If{$row \get $}
        \If{$a[col] \not= b[row]$}
            \State $cost\gets 1$
        \Else
            \State $cost\gets 0$
        \EndIf
        
        
    \EndFor
\EndProcedure
\end{algorithmic}
\end{algorithm}

3 algorithms

*1 fill_row

*2 fill_defaults

*3 main
